\documentclass[12pt]{article}

\usepackage[english]{babel}
\usepackage[utf8x]{inputenc}
\usepackage{amsmath}
\usepackage{amsfonts}
\usepackage{amssymb}
\usepackage{graphicx}
\usepackage[colorinlistoftodos]{todonotes}
\usepackage[margin=0.75in]{geometry}

\title{Discrete Math Assignment 4}
\author{Rachel Hwang}

\begin{document}
\maketitle

Collaborated with Andrew Ding \\
\begin{enumerate}
\item\emph{You throw a fair die 2013 times. What is the probability that the total is divisible by three?} \\
\\
The probability is $\frac{1}{3}$.\\

A single roll has a one-third probability of being divisible by 3, since two of six of the possible values have that property. Given the sum of any number $n$ of die rolls, the sum of our rolls $s(n)$ taken mod 3 will fall into one of three equivalence classes, $s(n) \mod 3 = 0$, $s(n) \mod 3 = 1$, $s(n) \mod 3 = 2$. No matter which of these three classes $s(n)$ falls into, there will be exactly 2 values of the $n+1$ th roll that will, added to $s(n)$, give us a number divisible by 3. If $s(n+1)$ in the first class, we need a roll of 3 or 6. If the second class, we need a roll of 2 or 5. If the third class, we need a 1 or 4. Since all rolls are independent events and no matter the value of $s(n)$, a third of the values will give us a values divisible by 3, the probability is one-third.
 
\item\emph{Alice and Bob are dealt ve cards each from the same 52-cards deck.
Calculate the probability that Alice gets a flush (five cards of the same suit) and Bob gets four of a kind. Are these two events independent?} \\
\\
The probability that Alice draws a flush is the number of ways she can [A: draw five cards of the same suit] divided by [B: the number of ways she can draw a five card hand]. A = the ways she can choose five cards from a single suit of thirteen cards, multiplied by the number of suits. B = the ways she can choose five cards from a deck of 52. This probability is 
\begin{align}
\nonumber \frac{\binom{4}{1} \cdot \binom{13}{5}}{\binom{52}{5}} &= \frac{12 \cdot 11 \cdot 10 \cdot 9}{51 \cdot 50 \cdot 49 \cdot 48}
\end{align}
Now, given that Alice has already drawn a flush, the probability that Bob draws a four of a kind can be thought of as [
\begin{align}
\nonumber \frac{\binom{8}{1} \cdot \binom{4}{4} \cdot (\binom{7}{1} \cdot \binom{1}{1} + \binom{12}{1} \cdot \binom{3}{1})}{\binom{47}{5}} &= \frac{8 \cdot 43 \cdot 5! \cdot 42!}{47!}
\end{align}
Let the probability that Alice draws her flush be $p(A)$ and the probability that Bob draws his four of a kind be $P(B)$. By the definition of conditional probability, $p(B|A)$ ($p(B)$ given $A$) is $\frac{p(A \land B)}{p(A)}$. Therefore, the probability that $p(A \land B) = p(B|A) \cdot p(A)$. So $p(A \land B)$ is
\begin{align}
\nonumber \frac{12 \cdot 11 \cdot 10 \cdot 9 \cdot 8 \cdot 43 \cdot 5! \cdot 42!}{51 \cdot 50 \cdot 49 \cdot 48 \cdot 47!} &= \frac{2}{4502365}
\end{align}
These events are \emph{not} independent because if either player gets their special hand, it effects the card pool of the other. For instance, if Alice has a flush, this means that Bob has only eight ranks from which to pick his four of a kind. Equivalently (order does not matter), if Bob has a four-of-a-kind, Alice only has 12 ranks in a single suit from which to pick her flush. $p(A \land B) \neq p(A) \cdot p(B)$.



\item\emph{Our class has 45 students in it, and its graduate version the next
door has an enrollment of 30. In our class, every student attends
any particular lecture with probability 70\% independently of other
students, and in Professor Babai's class two thirds of all lectures are
attended by everyone (and with probability 1/3 someone is missing).
Looking for you, your friend opens one of the two doors at random and
sees 30 students in the room. What is the probability that he opened
the right door?} \\
\\
Let be the probability of our friend seeing 30 students be called $p(T)$, the probability that he picks Razborov's classroom be $p(R)$, and the probability that he picks Babai's classroom be $p(B)$. We can calculate the probability of $p(T|R)$, since this is simply the probability that there will be 30 students in Razborov's classroom. By the theorem on Bernoulli trials, this is [the probability that 30 students show up] $\cdot$ [the probability that 15 students don't show up] $\cdot$ [ways to choose 30 students], since we are choosing some 30 students to have attend class, and multiplying by the probability that those 30 show and 15 do not. This is
\begin{align}
\nonumber p(T|R) = (.7)^{30} \cdot (.3)^{15} \cdot \binom{45}{30} &= \frac{45! \cdot (.7)^{30} \cdot (.3)^{15}}{15! \cdot 30!}
\end{align}
The probability of 30 students in Babai's classroom, $p(T|B)$, is $\frac{2}{3}$. We also know that prior to opening the door and seeing 30 students inside, the probability that our friend has picked our classroom (the right classroom) is simply 50\%, $p(R) = p(B) = .5$. We want to find $p(R|T)$, the probability that given that there are 30 students in the room, our friend picks our classroom. We have above determined the probability of $p(T|R)$ and $p(T|B)$, so we may determine $p(R|T)$ using Bayes Theorem:
\begin{align}
\nonumber P(R|T) = \frac{p(T|R)p(R)}{p(T | R)p(R) + p(T|B)p(B)} \approx .1433
\end{align}



\item\emph{Let $A$, $B$, $C$ be events in the same sample space such that every pair of them is independent and such that $p(A|B \land C) = p(A)$. Prove that
$p(B|A \lor C) = p(B)$.} \\
\\
Using the law of conditional probability, then the definition of independence, we know that
\begin{align}
\nonumber p(A|B \land C) = \frac{p(A \land (B \land C))}{(B \land C)} = \frac{p(A \land B \land C)}{(B \land C)} = \frac{p(A)p(B)p(C)}{p(B)p(C)}
\end{align}
Again, starting with the law of conditional probability, then using the distributive property of conjunction over disjunction, we know that 
\begin{align}
\nonumber &p(B|A \lor C) = \frac{p(B \land (A \lor C))}{(A \lor C)} = \frac{p((B \land A) \lor (B \land C))}{(A \lor C)}
\end{align} 
Using the inclusion/exclusion principle, then the definition of independence, this is equivalent to 
\begin{align}
\nonumber\frac{p((B \land A) \lor (B \land C))}{(A \lor C)} &= \frac{p((A \land B) + p(B \land C)) - p((A \land B) \land (B \land C))}{(A \lor C)} \\
\nonumber&= \frac{p(A)p(B) + p(B)p(C) - p(A \land B \land C))}{(A \lor C)}
\end{align}
As we showed above, the $p(A \land B \land C) = p(A)p(B)p(C)$ because of independence and the associative property. So,
\begin{align}
\nonumber&= \frac{p(A)p(B) + p(B)p(C) - p(A)p(B)p(C)}{(A \lor C)} \\
\nonumber&= \frac{p(B) \cdot (p(A) + p(C) - p(A)p(C)}{(A \lor C)}
\end{align}
Using the inclusion-exclusion principle again,
\begin{align}
\nonumber&= \frac{p(B) \cdot (p(A) \lor p(C))}{(A \lor C)} \\
\nonumber &= p(B)
\end{align}

\item\emph{You have one chance out of a million to win jackpot on a slot machine,
and you play it five million times in a row. Give a good estimate of
the chance that you will collect exactly five jackpot prizes.} \\
\\
Since, this a very rare event and a large number of trials, we can use the law of rare events, Poisson limit theorem. This gives us
\begin{align}
\nonumber \frac{n!}{(n-k)!\;k!}\cdot p^k(1-p)^{k-1} \approx e^{-\lambda}\cdot\frac{\lambda^k}{k!} = \frac{625}{24 \cdot e^5}\approx 0.175
\end{align}
where the number of trials $n$ is 5 million, the number of wins $k$ is 5, the probability $p$ is 1 in a million, and $\lambda = np = 5$.

\end{enumerate}



\end{document}
