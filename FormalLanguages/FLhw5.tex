\documentclass[a4paper]{article}

\usepackage[english]{babel}
\usepackage[utf8x]{inputenc}
\usepackage{amsmath}
\usepackage{amssymb}
\usepackage{amsthm}
\usepackage{graphicx}
\usepackage[colorinlistoftodos]{todonotes}
\usepackage[margin=0.75in]{geometry}
\usepackage{enumerate}

\title{Formal Languages Assigment 5}
\author{Rachel Hwang}

\begin{document}
\maketitle

\begin{itemize}
\item{Exercise 0.1}
\emph{Design a PDA to accept each of the following languages.}
\begin{enumerate}
\item $L$ = $\{a^ib^jc^k : i = j or j = k\}$
\\
\\
\\\\\\\\\\\\\\\\\\\\\\\\\\\\\\\\\\\\\\\\\\
\item $L$ = the language of all strings with twice as many 0's as 1's
\end{enumerate}
\newpage
\item{Exercise 0.2}
\emph{Show that if P is a PDA, then there is a PDA P2 with only two stack symbols, such that $L(P2) = L(P)$. Hint: Binary-code the stack alphabet of P.} \\
\\
Let $P = \{Q, \Sigma, \Gamma, \delta, q_0, Z_0\}$. We want to construct $P2 = \{Q', \Sigma, {0,1}, \delta', q_0', Z_0'\}$. \\
$Q'$ will contain $Q$, our $\Sigma$ is the same, and we select a stack alphabet of only two symbols. \\
\\
To define $\delta'$, let us first define some homomorphism $f:\Gamma \to {0,1}*$ encoding, where every symbol in $\Gamma$ gets a unique encoding. For example, we may number the states, and have the $f(q)$ simply be the binary representation of $q$'s number (state 1 $\to 0^{k-1}1$, etc.). The encoding will be $k = (log_2 n)$ rounded-up symbols long. Note that $Z_0'$ must be in {0,1}.\\
\\
Now, for every state $q \in Q$, letter $a \in \Sigma$, and symbol $X \in \Gamma$, for every transition $\delta(q,a,X) = (p,\gamma)$, assuming $f(X) = X_1\dots X_k$ where $X_i \in {0,1}$, we add to $Q'$, $k-1$ states. We are essentially adding one state and transitions between them to pop each $X_i$ off the stack, popping off all the $X_i$ which encode $X$ one digit at a time, and then finally pushing $\gamma$ onto the stack in a final transition to $p$.
For all $1 \leq i \leq k-2$, we add the transitions 
\begin{align*}
\delta(q, a, X_{1}) = (q_{1}, \epsilon) \\
\delta(q_i, \epsilon, X_{i+1}) = (q_{i+1}, \epsilon) \\
\delta(q_{k-1}, \epsilon, X_{k}) = (p, f(X)) \\
\end{align*}
We also add the start state $q_0'$, and the transition $\delta(q_0', \epsilon, Z_0') = (q_{0}, f(Z_0))$. \\
\\
We have now created a PDA $P2$ which has a stack alphabet of two characters and which manipulates the stack in a way equivalent to $P$. \\
\\
\\\\\\\\

\item{Exercise 0.3} \emph{Use the CFL pumping lemma to show each of these languages not to be context-free.}
\begin{enumerate}
\item $\{a^ib^jc^k : i < j < k\}$ \\
\\
For some number $n$, let us pick a string in $L,\;z = a^mb^{m+1}c^{m+2}$ where $m$ is larger than $n$ and $z = uvwxy$, $|vwz| \leq n$ and $vx \neq \epsilon$. By the pumping lemma, we can pick any $i$ and $uv^iwx^iy$ will be in $L$. \\
\\
We don't know where in the string $v$ and $w$ are, but it must be one of several cases \\
1) $v$ and $x$ are entirely in the $a$ section of $z$ \\
2) $v$ and $x$ are entirely in the $b$ section of $z$ \\
3) $v$ and $x$ are entirely in the $c$ section of $z$ \\
4) $v$ and $x$ are partially in the $a$ section of $z$, and partially in the $b$ section \\
5) $v$ and $x$ are partially in the $b$ section of $z$, and partially in the $c$ section \\
\\
Since $|vwz| \leq n < m$, we know that $v$ and $x$ cannot span all three sections. \\
\\
In case 1), we can pick $i = m+1$, adding at least one $a$. So the number of $a$'s is no longer less than $b$'s and $z' \notin L$. \\
In cases 2) and 4), we can pick $i = m+1$, adding at least one $b$. So the number of $b$'s is no longer less than $c$'s and $z' \notin L$. \\
In case 3), we can pick $i = 0$, subtracting at least one $c$. So number of $b$'s is no longer less than $c$'s and $z' \notin L$. \\
In case 5), we can pick $i = 0$, subtracting at least one $b$, so number of $a$'s is no longer less than $b$'s and $z' \notin L$. \\
\\
Now that we shown that for all cases, we can pick an $i$ such that  $z' \notin L$, by the pumping lemma, $L$ is not a context free language. \\

\item $L = \{a^nb^nc^i : i \leq n < k\}$ \\
\\
For some number $n$, let us pick a string in $L,\;z = a^{k-1}b^{k-1}c^{k-1}$ where $z = uvwxy$, $|vwz| \leq n$ and $vx \neq \epsilon$. By the pumping lemma, we can pick any $p$ and $uv^pwx^py$ will be in $L$. \\
\\
We don't know where in the string $v$ and $w$ are, but it must be one of several cases \\
1) $v$ and $x$ are entirely in the $a$ section of $z$ \\
2) $v$ and $x$ are entirely in the $b$ section of $z$ \\
3) $v$ and $x$ are entirely in the $c$ section of $z$ \\
4) $v$ and $x$ are partially in the $a$ section of $z$, and partially in the $b$ section \\
5) $v$ and $x$ are partially in the $b$ section of $z$, and partially in the $c$ section \\
\\
Since $|vwz| \leq n < m$, we know that $v$ and $x$ cannot span all three sections. \\
\\
In case 1), we can pick $p = m+1$, adding at least one $a$. So the number of $a$'s is now greater than $k$ and $z' \notin L$. \\
In case 2), we can pick $p = m+1$, adding at least one $b$. So the number of $b$'s is now greater than $k$ and $z' \notin L$. \\
In case 3), we can pick $p = m+1$, adding at least one $c$. So the number of $c$'s is now greater than $k$ and $z' \notin L$. \\
 \\
In cases 4), we can pick $p = m+1$, which will add at least $a$ and one $b$. So number of $a$'s and number of $b$'s are both no longer less than $k$'s and $z' \notin L$. \\
In cases 5), we can pick $p = m+1$, which will add at least $b$ and one $c$. So number of $b$'s and number of $c$'s are both no longer less than $k$'s and $z' \notin L$. \\
\\
Now that we shown that for all cases, we can pick an $i$ (the same $i$!) such that  $z' \notin L$, by the pumping lemma, $L$ is not a context free language. \\


\item $L =\{0^p : p\;is\; prime\}$ \\
\\
For some number $n$, let us pick a string in $L,\;z = 0^m$ where $m > n$ and is prime, and $z = uvwxy$, $|vwz| \leq n$ and $vx \neq \epsilon$. By the pumping lemma, we can pick any $i$ and $uv^iwx^iy$ will be in $L$. \\
\\
If we pick $i = m+1$, we get $z' = uv^{m+1}wx^{m+1}y$. We know that the length of $z'$ is $m+(|v|\cdot m) + (|x|\cdot m) = m + m\cdot(|v| + |x|)$. Since we are adding a multiple of our prime $m$ to the length of $z'$, the length of $z'$ is not prime. Therefore, by the pumping lemma, $L$ is not a context free language. \\
\\


\end{enumerate}

\item{Exercise 0.4} \emph{Give an algorithm to decide the following: Given a CFG G and one of its variables A, is there any sentential form in which A is the first symbol. Note: Remember that it is possible for A to appear first in the middle of some sentential form but then for all the symbols to its left to derive $\epsilon$.} \\
\\
Given $G = \{V, T, P, S\}$, we can build a directed graph $G = \{V,E\}$ where the set of vertices is the set of variables in $G$. Any two edges $v$ and $v'$ are connected, that is $(v, v') \in E$, iff $v$ appears in the leftmost position of a production rule $\Rightarrow v'$ appears in the leftmost position of a production rule.\\
\\
For all production rules in $G$, $X \to \beta_1 \dots \beta_k$ where $\beta_i \in \{V \cup R \cup \epsilon\}$, we perform the following steps: 
\begin{enumerate}
\item If $\beta_1 \in T$, then that production does not have a variable at its leftmost. Otherwise, add $(X, \beta_1)$ to $E$.
\item For each $\beta_i$, add $(X, \beta_i)$ to $E$ if \\
a) $\beta_i \notin T$ \\
b) for all $\beta_j$ where $j < i$, $\beta_j \notin T$ and $\beta_j$ is nullable. 
\end{enumerate}
Once we have have finished building our graph using the above method, we can perform a graph search to see if $S$ and $A$ are connected. If they are, this means that $S \Rightarrow^\ast A\beta_1 \dots \beta_k$, there is some setential form with $A$ in the left-most position.

\end{itemize}
\end{document}
